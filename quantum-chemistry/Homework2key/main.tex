%This is a LaTeX template for homework assignments
\documentclass{article}
\usepackage[utf8]{inputenc}
\usepackage{amsmath}
\usepackage{amsfonts} 
\usepackage{setspace}
\doublespacing
\begin{document}

\section*{Homework 2 Key}

\begin{enumerate}%starts the numbering

\item  
$\forall $ v in $ \mathbb{V}, \, |0\rangle$ is unique
\\ $|0'\rangle$ is another zero vector
\\ $|0\rangle = |0\rangle + |0'\rangle =|0'\rangle$ applying the zero identity property of the left term to the left side and applying the zero identity of the right term to the right side.
Unique because two different vectors but still preserves zero identity.
\item $\{\psi_1,\psi_2,\psi_3,\psi_4\}$ are orthogonal. Is $\Psi_a = c_1 \psi_1-c_2\psi_2$ and $\Psi_b=c_3\psi_3-c_4\psi_4$ orthogonal?
\\Yes! Since $\{\psi_1,\psi_2,\psi_3,\psi_4\}$ are orthogonal, the dot product of two $\psi$ terms in the set equals 0. 
\\ $\Psi_a\cdot\Psi_b=(c_1 \psi_1-c_2\psi_2) \cdot (c_3\psi_3-c_4\psi_4)$
\\ $\Psi_a\cdot\Psi_b=(c_1 \psi_1*c_3\psi_3 - c_1\psi_1*c_4\psi_4 - c_2\psi_2*c_3\psi_3+c_2\psi_2*c_4\psi_4)$
\\ $\Psi_a\cdot\Psi_b=0$ because the dot product of two $\psi$ terms equals 0
\item 
\begin{enumerate}
\item
Yes! a vector space. 
\\To prove it with these scalar multiplication axioms:
\\ - The scalar multiple of V by S, or $S\cdot V$ is in $\mathbb V$
\\ -  $(c+d)V=cV+dV$
\\ - $c(V_1+V_2)=cV_1+cV_2$
\\ $\forall~f \in V; S=\mathbb{R}; a,b \in S; f_1,f_2 \in f$
\\ $(a+b)f = af+bf$
\\ $(a+b)f''=af''+bf''$ Since a and b are scalar multiple of f'', af'' and bf'' are in vector space. 
\\ $a(f_1+f_2)=af_1+af_2$
\\ $a(f''_1+f''_2)=af''_1+af''_2$ Since a are scalar multiple of $f''_1$ and $f''_2$, $af''_1$ and $af''_2$ are in vector space.
\\ Therefore, any function within set V is a vector space.
\item Yes! A vector space
\\ \begin{math}
    s\left(\begin{pmatrix}
    a & b & c \\ 
    0 & d & e \\ 
    0 & 0 & f \\ 
    \end{pmatrix} 
    +\begin{pmatrix}
    g & h & i \\ 
    0 & j & k \\ 
    0 & 0 & l \\ 
    \end{pmatrix}\right) =
    \begin{pmatrix}
    s(a+g) & s(b+g) & s(c+i) \\ 
    0 & s(d+j) & s(e+k) \\ 
    0 & 0 & s(f+l)\\ 
\end{pmatrix}     \end{math}
\\ \\ a vector space because $M_{3x3}$ and $N_{3x3} \in \mathbb{V}$, $M_{3x3} + N_{3x3}\in \mathbb{V}$, and for every scalar multiple, $s$, of $M_{2x2}\in \mathbb{V}$, $sM_{3x3}\in \mathbb{V}$
\\ \\
     \begin{math}
    (s+s_1)\begin{pmatrix}
    a & b & c \\ 
    0 & d & e \\ 
    0 & 0 & f \\ 
    \end{pmatrix} =
    \begin{pmatrix}
    sa & sb & sc \\ 
    0 & sd & se \\ 
    0 & 0 & sf \\ 
    \end{pmatrix} 
    +\begin{pmatrix}
    s_1a & s_1b & s_1c \\ 
    0 & s_1d & s_1e \\ 
    0 & 0 & s_1f \\
    \end{pmatrix}
    \end{math} Both in vector space. Same logic as above.
\item Yes! A vector space
\\ \begin{math}
    s\left(\begin{pmatrix}
    a & b \\ 
    b & d \\ 
    \end{pmatrix} 
    +\begin{pmatrix}
    g & h \\ 
    h & j \\ 
    \end{pmatrix}\right) =
    \begin{pmatrix}
    s(a+g) & s(b+g) \\ 
    s(b+g) & s(d+j)\\ 

\end{pmatrix}     \end{math}
\\ \\ a vector space because $M_{2x2}$ and $N_{2x2} \in \mathbb{V}$, $M_{2x2} + N_{2x2}\in \mathbb{V}$, and for every scalar multiple, $s$, of $M_{2x2}\in \mathbb{V}$, $sM_{2x2}\in \mathbb{V}$
\\ \\
     \begin{math}
    (s+s_1)\begin{pmatrix}
    a & b \\ 
    b & d \\ 
    \end{pmatrix} =
    \begin{pmatrix}
    sa & sb\\ 
    sb & sd \\ 

    \end{pmatrix} 
    +\begin{pmatrix}
    s_1a & s_1b  \\ 
    s_1b & s_1d \\ 
    \end{pmatrix}
    \end{math} Both in vector space. Same logic as above.

\end{enumerate}
3.1 A basis is a set of linearly independent vectors that can be used as building blocks to make any other vector in the vector space. 
For 3b, the basis can be:
\begin{pmatrix}
1 & 0\\0 & 0\\
\end{pmatrix}
, 
\begin{pmatrix}
0 &1\\1 & 0 \\
\end{pmatrix}
, \begin{pmatrix}
0 &0\\0 & 1 \\
\end{pmatrix}
\\
a\begin{pmatrix}
1 & 0\\0 & 0\\
\end{pmatrix}
+ 
b\begin{pmatrix}
0 &1\\1 & 0 \\
\end{pmatrix}
+ c\begin{pmatrix}
0 &0\\0 & 1 \\
\end{pmatrix}
=\begin{pmatrix}
a &b\\b & c \\
\end{pmatrix}
where it is only true if $a=b=c=0$ so it is linear independent
\item 
\begin{enumerate}
\item
\\ let $|x\rangle = \sum_k x_k|b_k\rangle$
\\ let $|y\rangle = \sum_k y_k|b_k\rangle $;\quad $\langle y| =\langle b_k|\sum_k y_k$
\\ $\langle y|x \rangle = \langle b_k|\sum_k y_k^* \sum_k x_k|b_{k'}\rangle$
\\ $\langle y|x \rangle = \sum_k y_k^* \sum_k x_k \langle b_k|b_{k'}\rangle $
\\ $\langle y|x \rangle = \sum_k y_k^* \sum_k x_k \cdot \delta_{kk'}$
\\ $\langle y|x \rangle = \sum_k y_k^* x_k$ one summation is dropped
\item
\\ let $|x\rangle = \sum_k x_k|b_k\rangle$
\\ let $|y\rangle = \sum_k y_k|b_k\rangle $;\quad $\langle y| =\langle b_k|\sum_k y_k$
\\ $\langle y|x \rangle = \langle b_k|\sum_k y_k^* \sum_k x_k|b_{k'}\rangle$
\\ $\langle y|x \rangle = \sum_k y_k^* \sum_k x_k \langle b_k|b_{k'}\rangle $
\\ $\langle y|x \rangle = \sum_k y_k^* \sum_k x_k \cdot 1$
\\ $\langle y|x \rangle = \sum_k\sum_k y_k^* x_k$ 
\item
\\ let $|x\rangle = \sum_k x_k|b_k\rangle$
\\ let $|y\rangle = \sum_k y_k|b_k\rangle $;\quad $\langle y| =\langle b_k|\sum_k y_k$
\\ $\langle y|x \rangle = \langle b_k|\sum_k y_k^* \sum_k x_k|b_{k'}\rangle$
\\ $\langle y|x \rangle = \sum_k y_k^* \sum_k x_k \langle b_k|b_{k'}\rangle $
\\ $\langle y|x \rangle = \sum_k y_k^* \sum_k x_k \cdot (k+k')$
\\ $\langle y|x \rangle = (k+k')\sum_k\sum_k y_k^* x_k$ 
\end{enumerate}
\item
\\$f_n(x)=e^{inx}:n\in\mathbb{Z}$
\\Since problem did not indicate, I am spanning over all real numbers.
\\ m,n$\in\mathbb{Z}$
\\$\int_\infty^\infty e^{inx}e^{imx}$
\\$\int_\infty^\infty e^{ix(n+m)}$
\\ $2 \int_0^\infty e^{ix(n+m)}$
\\ $2e^{ix(n+m)}|_0^\infty$
\\ $2\frac{e^{ix(n+m)}}{i(n+m)}-e^0$
\\ If orthogonal, then: 
\\ $2\frac{e^{ix(n+m)}}{i(n+m)}-e^0=0$
\\ $\frac{e^{ix(n+m)}}{i(n+m)}-1=0$; you want left term to equal 1 to be orthogonal
\\ only orthogonal when function is defined over the $[0,2\pi]$ interval and m $\neq$ n
\item $\forall f \in \mathbb{V}$ that spans in [-1,1] interval using Taylor expansions, $f_1,f_2 \in \mathbb{V}$; $a, b \in \mathbb{V}$

Taylor expansion of f(x):
$f(x)=f(a)+f'(a)\frac{(x-a)}{1!}+f''(a)\frac{(x-a)^2}{2!}+...$

$a(f_1+f_2)(x)=f_1(a)+f'_1(a)\frac{(x-a)}{1!}+f''_1(a)\frac{(x-a)^2}{2!}+...+ f_2(a)+f'_2(a)\frac{(x-a)}{1!}+f''_2(a)\frac{(x-a)^2}{2!}+...$
\\$(a+b)f(x)=af(a)+af'(a)\frac{(x-a)}{1!}+af''(a)\frac{(x-a)^2}{2!}+...bf(a)+bf'(a)\frac{(x-a)}{1!}+bf''(a)\frac{(x-a)^2}{2!}+...$
\\in a vector space as using same axioms in Q3

\end{enumerate}%ends the numbering

\end{document}