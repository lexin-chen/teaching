%This is a LaTeX template for homework assignments
\documentclass{article}
\usepackage[utf8]{inputenc}
\usepackage{amsmath}
\usepackage{amsfonts} 
\usepackage{setspace}
\usepackage{amssymb}
\usepackage{amsmath,bm}
\usepackage{amsthm}
\usepackage{graphicx}
\usepackage{hyperref}
\hypersetup{
    colorlinks=true,
    linkcolor=blue,
    filecolor=magenta,      
    urlcolor=cyan,
    pdftitle={Overleaf Example},
    pdfpagemode=FullScreen,
    }
\urlstyle{same}
\doublespacing
\begin{document}

\section*{Homework 7 Key}
 
\begin{enumerate}%starts the numbering
    \item Calculate the average value of operators $x^2$ and $p_x^2$.
    \\ $x=\sqrt{\frac{\hbar}{2m\omega}}(b+b^\dagger)$
    \\ $\langle x\rangle=\langle\psi_n|\sqrt{\frac{\hbar}{2m\omega}}(b+b^\dagger)|\psi_n\rangle$
    \\ $\langle x\rangle=\sqrt{\frac{\hbar}{2m\omega}}\langle\psi_n|(b+b^\dagger)|\psi_n\rangle$
    \\ $\langle x^2\rangle=\frac{\hbar}{2m\omega}\langle\psi_n|(b^2+bb^\dagger+b^\dagger b+b^\dagger^2)|\psi_n\rangle$
    \\ $\langle x^2\rangle=\frac{\hbar}{2m\omega}\Big(\langle\psi_n|b^2|\psi_n\rangle +\langle\psi_n|bb^\dagger|\psi_n\rangle +\langle\psi_n|b^\dagger b|\psi_n\rangle+\langle\psi_n|b^\dagger^2|\psi_n\rangle\Big)$
    \\ $\langle\psi_n|b^2|\psi_n\rangle$ and $\langle\psi_n|b^\dagger^2|\psi_n\rangle$ equals 0 because the wavefunctions at different eigenstate are orthogonal hence a inner product of 0.
    \\ $\langle x^2\rangle=\frac{\hbar}{2m\omega}\Big(\langle\psi_n|bb^\dagger|\psi_n\rangle +\langle\psi_n|b^\dagger b|\psi_n\rangle\Big)$
    \\ $bb^\dagger = b^\dagger b +1$ based on its commutator relations, so
    \\ $\langle x^2\rangle=\frac{\hbar}{2m\omega}\Big(\langle\psi_n|N+1|\psi_n\rangle +\langle\psi_n|N|\psi_n\rangle\Big)$
    \\ $\langle x^2\rangle=\frac{\hbar}{2m\omega}\langle\psi_n|2N+1|\psi_n\rangle$
    \\ $\langle x^2\rangle=\frac{\hbar}{2m\omega}(2N+1)$
    \\ $\langle x^2\rangle=\frac{\hbar}{m\omega}(N+\frac{1}{2})$
    \\
    \\ $p=-i\sqrt{\frac{m\omega\hbar}{2}}(b-b^\dagger)$
    \\ $\langle p^2\rangle = -\frac{m\omega\hbar}{2}\langle\psi_n|(b^2-bb^\dagger-b^\dagger b+b^\dagger^2)|\psi_n\rangle$
    \\ $\langle p^2\rangle=-\frac{m\omega\hbar}{2}\Big(\langle\psi_n|b^2|\psi_n\rangle -\langle\psi_n|bb^\dagger|\psi_n\rangle -\langle\psi_n|b^\dagger b|\psi_n\rangle+\langle\psi_n|b^\dagger^2|\psi_n\rangle\Big)$
    \\ $\langle\psi_n|b^2|\psi_n\rangle$ and $\langle\psi_n|b^\dagger^2|\psi_n\rangle$ equals 0 because the wavefunctions at different eigenstate are orthogonal hence a inner product of 0.
    \\ $\langle p^2\rangle=-\frac{m\omega\hbar}{2}\Big(-\langle\psi_n|bb^\dagger|\psi_n\rangle -\langle\psi_n|b^\dagger b|\psi_n\rangle\Big)$
    \\ $bb^\dagger = b^\dagger b +1$ based on its commutator relations, so
    \\ $\langle p^2\rangle=\frac{m\omega\hbar}{2}\Big(\langle\psi_n|N+1|\psi_n\rangle +\langle\psi_n|N|\psi_n\rangle\Big)$
    \\ $\langle p^2\rangle=\frac{m\omega\hbar}{2}(2N+1)$
    \\ $\langle p^2\rangle=(m\omega\hbar)(N+\frac{1}{2})$ 
    \begin{enumerate}
    \item Calculate product of $\Delta x\Delta p$ for this state. 
    \\ $\Delta x\Delta p=\sqrt{\langle x^2\rangle-\langle x\rangle^2} \sqrt{ \langle p^2\rangle-\langle p\rangle^2}$
    \\ $\langle x^2\rangle=\frac{\hbar}{m\omega}(N+\frac{1}{2})$
    \\ $\langle x \rangle=0$ because odd function hence symmetry.
    \\ $\langle p^2\rangle=(m\omega\hbar)(N+\frac{1}{2})$ 
    \\ $\langle p \rangle=0$ because odd function hence symmetry.
    \\ $\Delta x\Delta p= \sqrt{\frac{\hbar}{m\omega}(N+\frac{1}{2})} \sqrt{(m\omega\hbar)(N+\frac{1}{2})}$
    \\ $\Delta x\Delta p= \hbar(N+\frac{1}{2})$
    \\ Recall the Heisenberg's uncertainty principle: $\Delta x\Delta p \ge \frac{\hbar}{2}$
    \\ $\hbar(N+\frac{1}{2}) \ge \frac{\hbar}{2}$; unless $N=0$ or ground state. 
    \end{enumerate}
\item A: 1,3,5-hexatriene (all trans isomer) and B: benzene. C-C bond length ($l=0.14nm$). A is 1D box with total length $L=6l$, B is circle with perimeter $L=6l$. 
    \begin{enumerate}
    \item Each system has 6 $\pi$ electron, calculate the energy difference between $\pi$ electrons in system A and B.
    \\ \underline{System A}: 6 $\pi$ electron, 2 $\pi$ electrons occupy each energy level, so 2 on n=1, 2 on n=2, and 2 on n=3.
    \\ $E=\frac{n^2h^2}{ml^2}$
    \\ $E=2*\frac{9h^2}{ml^2}+2*\frac{4h^2}{ml^2}+2*\frac{h^2}{ml^2}$
    \\ Substitute with $h=6.626*10^{34}$, $m=9.109 × 10^{-31}$, $l=6*0.14*10^{-9}$
    \\ $E = 2.39*10^{-18}J$
    
    \\ \underline{System B}: Degeneracy of 2 at every excited state $m_j=-1,1$
    \\ $E=\frac{m_j^2\hbar^2}{2I}$, $I=m(\frac{6l}{2\pi})^2$
    \\ $E=4*\frac{\frac{h}{2\pi}}{m(\frac{6l}{2\pi})^2}$
    \\ Substitute with $h=6.626*10^{34}$, $m=9.109 × 10^{-31}$, $l=6*0.14*10^{-9}$
    \\ $E= 1.37*10^{-18} J$
    \\ $\Delta E= 1.02*10^{-18} J$
    \item It could be used to predict aromatic stability because we predicted higher stability/lower energy in benzene system, which makes sense. 
    \item Some limitations include but not limited to: 1) these are 3D molecules that are approximated into 1D particle in a box or 2D rigid rotor. 2) Electron attraction/repulsion are not accounted for. 3) A fixed radius, which is fine for an approximation because very small fluctuations based on my knowledge. 
    \item The aromaticity rule that is supported by the 2D rigid rotor is the Huckel's rule that a cyclic molecule is aromatic if it has $4n+2$ $\pi$ electrons.
    \end{enumerate}
\item 
    \begin{enumerate}
    \item $[b^\dagger b, b]= b^\dagger bb-bb^\dagger b=(b^\dagger b-bb^\dagger )b$
    \\ $bb^\dagger=b^\dagger b+1$
    \\ $(b^\dagger b-bb^\dagger )b=(-1)b=-b$
    \item $[b^\dagger b, b^\dagger]= b^\dagger bb^\dagger-b^\dagger b^\dagger b=b^\dagger(bb^\dagger-b^\dagger b)=b^\dagger(1)=b^\dagger$
    \item $[b^\dagger b, bb^\dagger]$= $[N, N+1]=N(N+1)-(N+1)N=0$
    \end{enumerate}
\item $\forall n=0,1,2,...,$ $|n\rangle$ represent state s.t. $b^\dagger b|n\rangle=n|n\rangle$; $\langle n|n\rangle=1$
\\ $b^\dagger|n\rangle=\sqrt{n+1}|n+1\rangle$; $b|n\rangle=\sqrt{n}|n-1\rangle$ 
    \begin{enumerate}
        \item $\langle n|b^\dagger bbbb b^\dagger|n+2 \rangle, n=11$
        \\ $\langle 11|b^\dagger bbbb b^\dagger|13 \rangle$
        \\ $\sqrt{14}\langle 11|b^\dagger bbbb |14 \rangle$
        \\ $\sqrt{14}\sqrt{14}\langle 11|b^\dagger bbb |13 \rangle$
        \\ $\sqrt{13}\sqrt{14}\sqrt{14}\langle 11|b^\dagger bb |12 \rangle$
        \\ $\sqrt{12}\sqrt{13}\sqrt{14}\sqrt{14}\langle 11|b^\dagger b |11 \rangle$
        \\ $\sqrt{12}\sqrt{13}\sqrt{14}\sqrt{14}\langle 11|11 |11 \rangle$
        \\ $11*14\sqrt{12}\sqrt{13}$
        \\ $154\sqrt{156}$
        \item $\langle n|b^\dagger bbbb b^\dagger|n+2 \rangle, n=1$
        \\ $\langle 1|b^\dagger bbbb b^\dagger|3 \rangle$
        \\ $\sqrt{4}\langle 11|b^\dagger bbbb |4 \rangle$
        \\ $\sqrt{4}\sqrt{4}\langle 1|b^\dagger bbb |3 \rangle$
        \\ $\sqrt{3}\sqrt{4}\sqrt{4}\langle 1|b^\dagger bb |2 \rangle$
        \\ $\sqrt{2}\sqrt{3}\sqrt{4}\sqrt{4}\langle 1|b^\dagger b |1 \rangle$
        \\ $\sqrt{2}\sqrt{3}\sqrt{4}\sqrt{4}\langle 1|1 |1 \rangle$
        \\ $1*4\sqrt{2}\sqrt{3}$
        \\ $=4\sqrt{6}$
        \item $\langle n|\hat{p^3}|n \rangle =0$ 
        \\ $\langle n|n\rangle$ even, $\hat{p^3}$ odd. even * odd = odd, symmetry.
        \item $\langle n|\hat{p^3}|n+1 \rangle, n=13 $ 
        \\ Recall $\langle p^2\rangle = -\frac{m\omega\hbar}{2}\langle\psi_n|(b^2-bb^\dagger-b^\dagger b+b^\dagger^2)|\psi_n\rangle$ from #1
        \\ $\langle n|\hat{p^3}|n+1 \rangle = (-i)^3(\frac{m\omega\hbar}{2})^{\frac{3}{2}}\langle n|(b^2-bb^\dagger-b^\dagger b+b^\dagger^2)(b-b^\dagger)|n+1\rangle$
        \\ $= i(\frac{m\omega\hbar}{2})^{\frac{3}{2}}\langle n|(b^3-bb^\dagger b-b^\dagger bb+b^\dagger^2b- bbb^\dagger+bb^\dagger b^\dagger+b^\dagger bb^\dagger-b^\dagger^3)|n+1\rangle$
        \\ only terms with one rising and two lowering operator will lead to $\langle n|n\rangle$ because $(n+1)+1-1-1=n$. All others that does not lead to this, $\langle n|n\rangle$, will have inner product of 0.
        \\ $= i(\frac{m\omega\hbar}{2})^{\frac{3}{2}}\langle n|(-bb^\dagger b-b^\dagger bb- bbb^\dagger)|n+1\rangle$
        \\ $= -i(\frac{m\omega\hbar}{2})^{\frac{3}{2}}\langle n|(bb^\dagger b+b^\dagger bb+ bbb^\dagger)|n+1\rangle$
        \\ $= -i(\frac{m\omega\hbar}{2})^{\frac{3}{2}}\Big(\langle n|(bb^\dagger b|n+1\rangle + \langle n|(b^\dagger bb|n+1\rangle + \langle n|(bbb^\dagger|n+1\rangle\Big)$
        \\ $= -i(\frac{m\omega\hbar}{2})^{\frac{3}{2}}\Big(\sqrt{14}\langle 13|bb^\dagger |13\rangle + \sqrt 14\langle 13|b^\dagger b|13\rangle + \sqrt{15}\sqrt{14}\langle 13|bb|15\rangle\Big)$
        \\ $= -i(\frac{m\omega\hbar}{2})^{\frac{3}{2}}\Big(14\sqrt{14}\langle 13|13\rangle + 13\sqrt 14\langle 13|13\rangle + \sqrt{15}\sqrt{15}\sqrt{14}\langle 13|b|14\rangle\Big)$
         \\ $= -i(\frac{m\omega\hbar}{2})^{\frac{3}{2}}\Big(14\sqrt{14}\langle 13|13\rangle + 13\sqrt 14\langle 13|13\rangle + 15\sqrt{14}\langle 13|13\rangle\Big)$
         \\ $ = -i(\frac{m\omega\hbar}{2})^{\frac{3}{2}}*42\sqrt{14}$

        \item $\langle n|\hat{p^3}|n-3 \rangle, n=13 $ 


        \\ $= i(\frac{m\omega\hbar}{2})^{\frac{3}{2}}\langle n|(b^3-bb^\dagger b-b^\dagger bb+b^\dagger^2b- bbb^\dagger+bb^\dagger b^\dagger+b^\dagger bb^\dagger-b^\dagger^3)|n-3\rangle$
        \\ only terms with 3 rising will lead to $\langle n|n\rangle$ because $(n-3)+1+1+1=n$. All others that does not lead to this, $\langle n|n\rangle$, will have inner product of 0.         
        \\ $= i(\frac{m\omega\hbar}{2})^{\frac{3}{2}}\langle n|-b^\dagger^3|n-3\rangle$
        \\ $= -i(\frac{m\omega\hbar}{2})^{\frac{3}{2}}\sqrt{11}\langle 13|b^\dagger^2|11\rangle$   
        \\ $= -i(\frac{m\omega\hbar}{2})^{\frac{3}{2}}\sqrt{11}\sqrt{12}\langle 13|b^\dagger|12\rangle$  
        \\ $= -i(\frac{m\omega\hbar}{2})^{\frac{3}{2}}\sqrt{11}\sqrt{12}\sqrt{13}\langle 13|13\rangle$   
        \\ $= -i(\frac{m\omega\hbar}{2})^{\frac{3}{2}}\sqrt{1706}$
    \end{enumerate}
\item Take $\psi^2$, you have $e^{something}$ are opposite signs so e part becomes 1, because $m_j$ consists of both positive and its negative counterparts. You get that it doesn't depend on $\phi$, but instead $\theta$.

\end{enumerate}%ends the numbering

\end{document}